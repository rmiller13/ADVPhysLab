\documentclass[twocolumn]{article}
\usepackage{graphicx}
\usepackage{mathtools}
\begin{document}  

\begin{titlepage}

\centering

\Huge Modern Interferometry Lab \\[.5cm]

\large A. Ford \\ A. Garcia \\ R. Miller 

\date{\today}

%\end{center}

\end{titlepage}
 
 \tableofcontents

%\listoffigures


\section{Background}  

\section{Materials}
		
\section{Procedure}
		
\section{Experiments}
	\subsection{Experiment 1}
	\subsection{Experiment 2}
\appendix
\section{Definitions}

\textbf{Piezo}:
    A piezo actuator consists of ceramic layers stacked together. These ceramic pieces convert electrical energy into movement. The piezo is used to make very small and accurate displacements in the nano- and micro-positioning.

\textbf{Lenses}:
    Lenses were placed between the laser and the first steering mirror to expand the width of the laser beam. By expanding the laser beam width we are able to see the fringes more clearly and align the interferometer more easily.
    
%
\textbf{Visibility}:
    Visibility is used to see how well aligned the interferometer is. The formula for visibility is:

     \begin{equation}\label{eqn:Visibility}  
        \gamma= \frac{I_(Max) - I_(Min)}{I_(Max) + I_(Min)}
     \end{equation}    
%
$I_(Max)$ is the maximum intensity that is recorded for the interferometer while $I_(Min)$ is the lowest intensity that is recorded.  If $\gamma$=1 then the interferometer is perfectly aligned, but if $gamma$=0 then there is no alignment. 

\section{Citations}

 \end{document}