\documentclass[twocolumn]{article}
\usepackage{graphicx}
\usepackage{mathtools}
\begin{document}  

\begin{titlepage}

\centering

\Huge Modern Interferometry Lab \\[.5cm]

\large A. Ford \\ A. Garcia \\ R. Miller 

\date{\today}

%\end{center}

\end{titlepage}
 
 \tableofcontents

%\listoffigures


\section{Background}  

\section{Materials}
		
%\section{Procedure}
\section{Procedure}

\subsection{Inspection of Equipment}

To ensure proper operation of the interferometer, all optics must be in good
shape, and should be inspected prior to setup. Also note that latex gloves
should be worn at all times when handling optics directly, and care should be
taken when handling mounted optics.

When inspecting the optics, check that they are reasonably free of dust, oil,
and fingerprints. If there are fingerprints or large quantities of dust or oil,
then ask the instructor for assistance in cleaning the optics.

After checking the optics, be sure to verify the functionality of the
Helium-Neon laser. Please note that some He-Ne lasers have a warm up period, and
may also have a shutter that needs to be opened.

\subsection{Preliminary Beam Alignment}

\begin{enumerate}
    \item Mounting the He-Ne Laser
        \begin{enumerate}
        \item Insert the laser into the ULM-TILT before attaching optical
        posts and locate a position near the corner of the breadboard where the
        laser path lies over a row of screw holes.
        \item Mount the ULM-TILT base to the breadboard with optical posts and
        post holders. As one screw hole lies below the laser mount, the laser
        must be removed for this step.
        \item Re-insert the laser and tighten the set screw to securely hold the
        laser.
        \item Verify that the laser's beam still lies over the row of holes.
        \end{enumerate}
    \item Mounting the Silvered Mirrors

        \emph{Skip if your mirrors are already mounted to a Kinematic Mirror
        Mount}

        Note: You may have to remove the mirror collar (retaining ring) to
        properly mount the mirror holder to the Kinematic Mirror Mount.

        \begin{enumerate}
        \item Attach the 1" mirror holder to the Kinematic Mirror Mount
        \item Carefully place the mirror into the holder and tighten the mirror
        collar.
        \item Check that the mirror is not loose in the holder. If it is, cut
        several small squares of paper (a little less than the mirror's
        diameter) and layer them in the mirror holder until the mirror is snug.
        \end{enumerate}
    \item Laying out the Beam Path

\end{enumerate}
        


		
\section{Experiments}
	\subsection{Experiment 1}
	\subsection{Experiment 2}
\appendix
\section{Definitions}

\textbf{Piezo}:
    A piezo actuator consists of ceramic layers stacked together. These ceramic pieces convert electrical energy into movement. The piezo is used to make very small and accurate displacements in the nano- and micro-positioning.

\textbf{Lenses}:
    Lenses were placed between the laser and the first steering mirror to expand the width of the laser beam. By expanding the laser beam width we are able to see the fringes more clearly and align the interferometer more easily.
    
%
\textbf{Visibility}:
    Visibility is used to see how well aligned the interferometer is. The formula for visibility is:

     \begin{equation}\label{eqn:Visibility}  
        \gamma= \frac{I_(Max) - I_(Min)}{I_(Max) + I_(Min)}
     \end{equation}    
%
$I_(Max)$ is the maximum intensity that is recorded for the interferometer while $I_(Min)$ is the lowest intensity that is recorded.  If $\gamma$=1 then the interferometer is perfectly aligned, but if $gamma$=0 then there is no alignment. 

\section{Citations}

 \end{document}
