\paragraph{}
	Have you ever wondered about the colors that can be seen on soap bubbles or when you mix oil and water? The reason that you can see various colors is due to the light waves traveling through them interfering with each other. 

	When the light waves interfere it will be either constructive or destructive interference; constructive allows them to add amplitudes while destructive reduces amplitudes or cancels them out completely.

	Another example of interference would be the splitting of white light into the various colors of the visible light spectrum using a diffraction grating. The various colors you see are the product of constructive interference, while the dark patches are destructive.

    An instrument that measures the intensity of the wave interference is called an interferometer. Interferometers are used in various fields for accurate measurements due to their accurate measurements. One of the simplest models for an interferometer is the Michelson Interferometer.

	With a Michelson interferometer you can measure distances and refractive indices of test materials introduced into the beam path. To make the measurements remember that light is an electromagnetic wave that can be described by the scalar equation

