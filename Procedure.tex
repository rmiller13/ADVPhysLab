\section{Procedure}
\label{sec:procedure}

\subsection{Inspection of Equipment}
\label{sub:inspection}

To ensure proper operation of the interferometer, all optics must be in good
shape, and should be inspected prior to setup. Also note that latex gloves
should be worn at all times when handling optics directly, and care should be
taken when handling mounted optics.

When inspecting the optics, check that they are reasonably free of dust, oil,
and fingerprints. If there are fingerprints or large quantities of dust or oil,
then ask the instructor for assistance in cleaning the optics.

After checking the optics, be sure to verify the functionality of the
Helium-Neon laser. Please note that some He-Ne lasers have a warm up period, and
may also have a shutter that needs to be opened.

\subsection{Preliminary Beam Alignment}
\label{sub:preliminary}

    \begin{enumerate}
    \item Mounting the He-Ne Laser
        \begin{enumerate}
        \item Insert the laser into the ULM-TILT before attaching optical
        posts and locate a position near the corner of the breadboard where the
        laser path lies over a row of screw holes.
        \item Mount the ULM-TILT base to the breadboard with optical posts and
        post holders. As one screw hole lies below the laser mount, the laser
        must be removed for this step.
        \item Re-insert the laser and tighten the set screw to securely hold the
        laser.
        \item Verify that the laser's beam still lies over the row of holes.
        \end{enumerate}
    \item Mounting the Silvered Mirrors

        \emph{Skip if your mirrors are already mounted to a Kinematic Mirror
        Mount}


        \begin{enumerate}
        \item Attach the 1" mirror holder to the Kinematic Mirror Mount

        \emph{Note}: You may have to remove the mirror collar (retaining ring)
        to properly mount the mirror holder to the Kinematic Mirror Mount.
        \item Carefully place the mirror into the holder and tighten the mirror
        collar.
        \item Check that the mirror is not loose in the holder. If it is, cut
        several small squares of paper (a little less than the mirror's
        diameter) and layer them in the mirror holder until the mirror is snug.
        \end{enumerate}
    \item Laying out the Beam Path
        \begin{enumerate}
        \item Place Mirrors 1 and 2, and the Piezo Mirror as indicated by
        Fig.~\ref{fig:interferometer}. These will form the primary beam path.
        \item To ease alignment, attempt to direct the beam so that it lies over
        the grid created by the breadboard holes.
        \item Align the mirrors so that the beam strikes the center of each
        mirror. Rough adjustments should be done by raising, lowering, or
        rotating the post. Fine adjustments should be done using the adjustment
        screw.
        \end{enumerate}
    \item Leveling the Beam
    \label{step:leveling}
        \begin{enumerate}
        \item Place the iris halfway between the first two mirrors and shine the
        beam through the iris. If the laser reflected off of the
        second mirror does not retro-reflect through the iris onto the first mirror,
        adjust the mirrors until it does.
        \item Repeat this with the other mirrors until the entire beam path is
        level.
        \item When completed, the beam should travel from the He-Ne laser to the
        Piezo mirror, then retro reflect back to the laser, aligned as close to
        the aperture as possible.
        \end{enumerate}
    \item Introducing the Beam Splitter Cube
        \begin{enumerate}
		\item How to mount the Beam Splitter Cube.
        \item Mount the Beam Splitter Cube on the breadboard between
			  Mirror 2 and the Piezo Mirror. The Beam Splitter should split the laser beam perpendicular to the bean path.
        \item Using the iris, verify that the beam is level between Mirror 2 and
        the Piezo Mirror. If it is not, you may need to use the alignment screws
        on the Kinematic Platform to level the Beam Splitter. You \emph{should
        not} have to adjust the mirrors, as they were level previously.
        \label{step:beamsplitter_level}
        \end{enumerate}
    \item Introducing Mirror 3 and Lens 3
        \begin{enumerate}
        \item Place Mirror 3 perpendicular to the beam path between Mirror 2 and
        the Piezo Mirror, and an equal distance from the beam splitter as the
        Piezo Mirror. Refer to Fig.~\ref{fig:interferometer} for clarification.
        \item Verify that the center of Mirror 3 is the same height as that of
        Mirrors 1 and 2, and all other optics.
        \item Place Lens 3 perpendicular to the beam path between Mirror 2 and
        the Piezo Mirror, opposite from Mirror 3. 
        \item Verify that the center of Lens 3 is the same height as that of all
        other optics.
        \item Now using the iris, verify that the beam between Mirror 3 and Lens
        3 is level. Again, if it is not level, use the alignment screws on the
        Kinematic Platform for adjustment, as in
        Step~\ref{step:beamsplitter_level}.  
        \end{enumerate}
    \end{enumerate}
        
\subsection{Further Beam Alignment}
\label{sub:further}

    %\begin{enumerate}
    %\item First Visualization
    \subsubsection{First Visualization}
        \begin{enumerate}
        \item Place a paper or screen vertically after Lens 3.
        \item Depending on how well aligned your interferometer is, you should
        see either two distinct beams, or a single beam with dark banding. If
        you see two beams, use the alignment screws on Mirror 3 and the Piezo
        Mirror to align these two beams on screen.

            \emph{Alignment Tips}:
            \begin{itemize}
            \item If you see purely vertical banding, your misalignment lies
            only in the horizontal direction.
            \item Similarly, if you see purely horizontal banding, your
            misalignment lies only in the vertical direction.
            \item Diagonal banding is a combination of both vertical and
            horizontal banding.
            \item Curved banding is a good sign, as the misalignment is now
            minimal.  You should seek the center of curvature of the bands.
            \end{itemize}
        \end{enumerate} 
    %\item Test your interference
    \subsubsection{Test your interference}

        The interferometer should be sensitive enough to pick up vibrations in
        the table and breadboard generated by tapping. These would be seen as
        dark bands moving on the screen.
    %\item Achieving Optimum Focus
    \subsubsection{Achieving Optimum Focus}
        \begin{enumerate}
        \item Set the centerline height of Lenses 1 and 2 to match all other
        optics.
        \item Place Lenses 1 and 2 into the beam path between the He-Ne laser
        and Mirror 1, and adjust their position to achieve the best focus
        visible on the screen.
        \item Mount the post holders to the board and verify that the beam path
        is still level throughout the entire interferometer. (Refer to Section~\ref{sub:preliminary} Step~\ref{step:leveling} if needed)
        \end{enumerate}
    %\end{enumerate}


\subsection{Using the Photo Diode}
\label{sec:photodiode}

    \subsubsection{Introducing the Photo Diode}
        \begin{enumerate}
        \item Replace Lens 3 with the Photo Diode and center the beam on the
        sensor.
        \item Connect the Photo Diode to CH1 of the oscilloscope using a BNC
        cable and the Variable Terminator set to the closest match possible to
        your oscilloscope's impedance.
        \item Turn on the voltage bias.
        \item Now fringes can be seen as fluctuations in the signal.
        \end{enumerate}
