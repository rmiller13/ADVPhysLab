	The purpose of this experiment is to see how an interferometer works. The Michelson interferometer is the design that 
will be used for the experiment. A Michelson interferometer is a simple interferometer that is used to make very small and precies
measurements, in fact the Michelson interferometer is the layout used in LIGO. A Michelson interferometer works by having a laser 
shoot down a path into a beam splitter that splits the laser beam into two seperate paths. At the end of each path there is a mirror
that will reflect the laser back toward the beam splitter which then combines the two beams. After the recombination the lasers 
will be either in or out of phase, due to the superposition principle. The superposition principle basically says that when two waves
pass through the same point in space their amplitudes will combine; this causes the net amplitude to add or subtract, known as
constructive and destructive interference respectively. Constructive interference happens if the waves are in phase while destructive 
interference happens when the waves are out of phase.

\emph{Insert picture of constructive and destructive interference?}

	In this experiment the only way that the phase will change will be due to the distance between the mirrors and the beam splitter.
The phase difference,$\Delta \phi$, is calculated by 
	\begin{align}
		\nonumber \Delta \phi &=\frac{2\pi}{\lambda} \Delta p
	\end{align}
with $\Delta p$ being the difference in the paths and $\lambda$ being the wavelength of the laser being used. From there we can then say that 
	\begin{align}
		\nonumber \Delta p &=\sum(n_2 d_2)-\sum(n_1 d_1) \\
		\nonumber		&=n_2 d_2 - n_1 d_1
	\end{align} 
where $d$ is the distance from the beam splitter to the cube and $n$ is index of refraction of the medium the beam is traveling 
through. For this experiment the medium's are the same, they are both traveling through air, so then $n_1=n_2$ which results in
	\begin{align}
		\nonumber \Delta p &= n_2 d_2 - n_1 d_1 \\
		\nonumber 	&= n(d_2 - d_1) \\
		\nonumber \Delta \phi &=\frac{2\pi n}{\lambda} (d_2 - d_1)
	\end{align}

	The change in phase will then affect the intensity of the beam, causing fringes to appear. These fringes will be red and black
rings on a screen or piece of paper behind the lens. The black rings signify a destructive interference while the red rings are 
due to the constructive interference. 

	Using a photodiode to get the fringes to appear on an oscilloscope it is then possible to get the voltage required to move 
from one fringe to another. Using this information with the properties of the piezo stack it is possible to get the wavelength of
the laser. 


